\documentclass[a4paper,11pt]{article}

\usepackage[utf8]{inputenc}
\usepackage[T1]{fontenc}
\usepackage[spanish]{babel}
\usepackage[pdftex]{graphicx}
\usepackage[hyphens]{url}
\usepackage[pdftex]{hyperref}
\hypersetup{
	pdftitle={Reglamento Velocistas},		% title
	pdfauthor={Club de Robótica FIUBA},	% author
	pdfsubject={Club de Robótica FIUBA},	% subject of the document
	colorlinks=true,		% false: boxed links;
	citecolor=black,		% color of links to bibliography
	filecolor=black,		% color of file links
	linkcolor=black,		% color of internal links
	urlcolor=black			% color of external links
}

%\usepackage{iwona}
\parskip=3pt
\headheight = 62pt

\usepackage{anysize}
%\marginsize{izquierdo}{derecho}{superior}{inferior}
\marginsize{2cm}{2cm}{1cm}{1cm}


\usepackage{fancyhdr}
\usepackage{lastpage}

%\fancyhf{} % clear all header and footer fields
\fancypagestyle{plain}{%
\fancyhead{} % get rid of headers
\renewcommand{\headrulewidth}{1pt} % and the line
\lhead{\includegraphics[height=2.4cm]{logoclub}}
%\chead{}
\rhead{\includegraphics[height=2cm]{logofiuba.jpg}}
\lfoot{ www.clubderobotica.com.ar}
\cfoot{}
\rfoot{ \thepage \ de \pageref{LastPage}}
}

\pagestyle{plain}

\let\oldenumerate\enumerate
\renewcommand{\enumerate}{
  \oldenumerate
  \setlength{\itemsep}{1pt}
  \setlength{\parskip}{0pt}
  \setlength{\parsep}{0pt}
}


\let\olditemize\itemize
\renewcommand{\itemize}{
  \olditemize
  \setlength{\itemsep}{1pt}
  \setlength{\parskip}{0pt}
  \setlength{\parsep}{0pt}
}

\usepackage{transparent}
\usepackage{eso-pic}
\newcommand\BackgroundPic{
  \put(0,0){
    \parbox[b][\paperheight]{\paperwidth}{%
      \vfill
      \centering
      {\transparent{0.05}\includegraphics[width=0.9\paperwidth]{logoclub}}
      \vfill
    }
  }
}


\newcommand{\cm}{\ensuremath{\mbox{~cm}}}


\author{Club de Robótica \\ Facultad de Ingeniería \\ Universidad de Buenos Aires}
\title{Competencia de Robótica 2014}


\begin{document}

\AddToShipoutPicture{\BackgroundPic}

\begin{center}
  {\Huge \textbf{Club de Robótica FIUBA}}
  \vspace{0.5cm}

  {\huge Competencia de Robótica 2014}
\end{center}

\section*{Objetivo de la competencia ``Velocistas''}
El objetivo de la competencia es diseñar y construir un robot seguidor de líneas autónomo, que complete un circuito preestablecido en el menor tiempo posible.
\section*{Organización}
\subsection*{General}
\begin{itemize}
  \item La organización se reserva el derecho de introducir cualquier cambio en la normativa, cuando lo estime oportuno para el desarrollo de las pruebas.
  \item El jurado se conformará de 2 personas seleccionadas por los organizadores de la competencia.
  \item Las decisiones de los jueces serán, en todo momento, inapelables.
  \item Pueden participar de esta competencia cualquier persona interesada. De ser menor de 18 años debe asistir acompañado por un mayor responsable.
  \item Los organizadores se reservan el derecho de admisión. En caso de conductas inapropiadas, a criterio del jurado, los organizadores podrán excluir a los equipos involucrados.
\end{itemize}

\subsection*{Inscripción}
\begin{itemize}
  \item Cada robot podrá ser registrado (a través del formulario correspondiente) por un equipo de hasta 4 miembros.
  \item Cada robot llevará un nombre. En caso de que dos robots sean registrados con el mismo nombre, la prioridad está determinada por el orden de pre-inscripción. Los restantes equipos podrán seleccionar otro nombre, o simplemente agregar un identificador (por ejemplo: robot\_2).
  \item Cada equipo debe tener al menos 1 miembro mayor de 18 años, que será responsable por los miembros menores de edad que pueda tener el equipo.
  \item En el día de la competencia, por la mañana será la confirmación de asistencia y verificación de robots (``inscripción definitiva''). Es obligatorio presentarse antes de la finalización de este período para ser incluido en el torneo. Una vez cerrada la inscripción definitiva, se arma el cronograma de la competencia y orden de turnos, con todos los inscriptos.
  \item El horario de cierre de la ``inscripción definitiva'' se definirá en los días previos a la competencia. El horario de comienzo de la competencia se determinará el día del evento.
  \item Una vez publicado el orden, cada equipo es responsable de estar presente en el momento que corresponda su turno para competir.
  \item Parte de la calificación obtenida por los robots consiste en la documentación de cada robot debe presentar (ver ``El robot''). \textbf{Esta documentación debe ser enviada hasta una semana antes de la competencia}, de forma tal que el jurado tenga el suficiente tiempo para evaluarla.
  \item Además, esta información será publicada luego de finalizar la competencia, con el objetivo de favorecer el aprendizaje y fomentar el desarrollo de nuevos robots. Al realizar la inscripción del robot, todos los miembros del equipo están aceptando este compromiso.
\end{itemize}

\section*{El robot}
\subsection*{Requerimientos mínimos que debe cumplir el robot:}
\begin{itemize}
  \item El robot debe poseer como máximo $10\cm$ de alto ya que podrá haber un túnel de 10 cm de altura mínima.
  \item El robot debe poseer como máximo $22\cm$ de ancho, ya que es el ancho de la largada.
  \item El robot no puede tener ningún tipo de material o elementos que puedan dañar la pista.
  \item Cada robot debe tener un interruptor (\emph{switch}) que permita detenerlo inmediatamente. El interruptor debe ser visible y accesible quedando a criterio de los jueces el cumplimiento de este requerimiento.
  \item El robot debe ser completamente autónomo, es decir, no podrá necesitar de ningún tipo de conexión o comunicación con el exterior para realizar la competencia. Sí está permitido que el robot transmita datos útiles para el análisis de su desempeño. En caso de ser solicitado por el jurado, el equipo deberá demostrar que el robot puede funcionar sin este enlace activado.
  \item El robot deberá utilizar baterías. Está prohibido el uso de cualquier tipo de combustible.
\end{itemize}

No hay más restricciones, se pueden usar kits de robótica, kits de electrónica, o diseños completamente propios.

Se puede utilizar cualquier procesador o circuito para controlar el auto. El mismo criterio se aplica a los sensores, donde cualquiera está permitido.

Si bien no existen limitaciones para el largo del robot, y las referidas al ancho del robot son bastante laxas, se recomienda tener en cuenta las dimensiones de la pista ya que, por ejemplo, un robot muy grande quizás no pueda girar en las curvas más pronunciadas, o se caería fácilmente de la pista.

\subsection*{Documentación del robot}
Parte de la calificación del robot comprende la evaluación, por parte del jurado, de la documentación prevista por el equipo.
Dicha documentación debe incluir:
\begin{enumerate}
  \item Carátula: documento editable provisto por el Club (descargar en la página) 
  \item Introducción:
  \begin{itemize}
    \item Descripcion basica del funcionamiento del robot 
    \item Objetivos (simpleza, excelencia, económico, reciclado, repetible, etc.)
  \end{itemize}
  \item Mecánica:
  \begin{itemize}
    \item Descripción de la estructura mecánica
    \item Especificaciones técnicas (Tipo, potencia y rpm de los motores, fuente de alimentación, etc.)
  \end{itemize}
  \item Electrónica:
  \begin{itemize}
    \item Descripción del circuito y mención de los integrados utilizados
    \item Esquemático y/o PCB 
  \end{itemize}
  \item Programación:
  \begin{itemize}
    \item Método de programación y programador utilizado
    \item Descripción de la lógica del código y lenguaje utilizado 
  \end{itemize}
  \item Conclusiones: Conclusiones del trabajo, costo total del robot, posibles mejoras a implementar, alternativas consideradas, etc.
  \item Anexo:
  \begin{itemize}
    \item Mecánica: Diagrama y planos (Diagramas 3d, dibujos, fotos del ensamblaje y/o croquis)
    \item Electrónica: Diagrama en bloques (optativo)
    \item Programación: 
    \begin{enumerate}
      \item Código fuente (en caso de que sea analógico, especificarlo, y explicar como calibraron el robot)
      \item Diagrama de flujos (optativo)
    \end{enumerate}
  \end{itemize}
\end{enumerate}

Las descripciones deben ser breves y concisas. La documentación no debe exceder el límite de 6 hojas máximo sin contar el anexo. 

El formato del documento deberá ser: 

Hoja: A4, Letra: Arial, Tamaño: 11, Alineación: Justificada, Interlineado: Sencillo y Sangría: $0.5\cm$. 

Se deberá enviar el archivo en formato PDF.

Un jurado puede solicitar a algunos participantes hacer una breve exposición oral e informal de los diseños luego de finalizar la carrera, ante toda la audiencia.

El objetivo de publicar los diseños y solicitar las presentaciones orales es favorecer el aprendizaje de todos los concursantes, estudiando los diseños de los demás.

Al momento de la inscripción, los miembros del equipo aseguran que la información presentada es de su propiedad intelectual, y/o los debidos créditos fueron incluidos. También acuerdan ceder los derechos de publicación de la información a los organizadores del evento, siendo ésta debidamente referenciada (es decir, aceptan que la información sea publicada en la página del club de robótica, indicando quiénes son los autores).

\section*{El circuito}
\begin{itemize}
  \item La pista de competencia consta de un circuito cerrado con una línea blanca de $2.0\cm$ ($\pm 0.2\cm$) de ancho sobre un fondo negro.
  \item El fondo negro es pintura mate sobre una tabla de madera, y la línea blanca se forma con pintura sintética blanco mate.
  \item El ancho de la pista es de $22.0\cm$ ($\pm0.5\cm$).
  \item La largada es un arco de $22\cm$ de ancho y $10\cm$ de alto.
  \item La pista tendrá una elevación con respecto al nivel del suelo en todo su recorrido.
  \item El circuito contará con curvas de un radio de curvatura mínimo de $20.0\cm$ ($\pm 0.5\cm$) con peralte nulo.
  \item El circuito podrá incluir un puente con pendiente máxima de 20 grados. El delta de pendiente máximo es 1 grado/cm.
  \item El circuito podrá incluir un túnel con altura mínima de $10\cm$.
  \item El circuito no tiene una longitud máxima.
  \item El circuito podrá tener cruces (como un 8). El ángulo mínimo de los cruces es 60 grados.
  \item El circuito se conocerá el día del evento.
\end{itemize}

\section*{La carrera}
Habrá 3 pistas. El circuito oficial que corresponde a la pista descripta en
``El circuito'', y dos pistas de uso libre por los participantes para probar el
robot antes o durante sus ``tiempos de pista'' (ver sección ``Formato''). Las
dos últimas son de la misma conformación que el circuito original, aunque de
menor longitud y menor cantidad de curvas. Las pistas de prueba incluirán por
lo menos una curva con el radio de curvatura más chico utilizado en el circuito
oficial.

\subsection*{Formato}
\begin{itemize}
  \item Cada equipo dispone de 3 minutos de ``tiempo en pista'' en su primer turno. Este tiempo incluye el tiempo utilizado para hacer los ajustes permitidos (ver abajo) durante la carrera. El equipo puede intentar completar el circuito todas las veces que lo desee dentro de los 3 minutos de su turno.  El mejor tiempo de vuelta de ese turno quedará registrado.
  \item Cada equipo tiene asegurado por lo menos 2 turnos. En función de la cantidad de participantes, podrán asignarse más turnos por equipo.
  \item A partir del segundo turno, el ``tiempo en pista'' es de 1 minuto y medio.
  \item Después de un minuto de haber sido convocados, el reloj comenzará a contar el ``tiempo en pista'', aunque el equipo no haya comenzado a correr.
  \item Al momento de ser convocado el equipo puede decidir pasar su turno. En ese caso se le asignará un ``tiempo de pista reducido'' de 1 minuto de duración al final de la ronda actual. El equipo tiene garantizado un descanso de 5 minutos mínimo entre el turno salteado y el nuevo. Salvo por la duración el ``tiempo de pista reducido'' no tiene ninguna diferencia con el regular.
\end{itemize}    

\subsection*{Durante la carrera}
\begin{itemize}
  \item El equipo puede detener el robot y volver a comenzar el circuito todas las veces que lo desee.
  \item Se le permitirá al robot completar el circuito, aún si excediera los 3 minutos totales, siempre y cuando haya iniciado la vuelta dentro de los 3 minutos correspondientes.
  \item Se pueden hacer ajustes entre carreras. Ninguna acción realizada sobre el robot, detiene el ``tiempo en pista''.
  \item Se computará como una vuelta completa, cuando el robot pase por la largada y realice el recorrido completo sin ser asistido en forma externa. Cuando el robot pasa por la largada, interrumpirá una barrera infrarroja y se comenzará a contar el tiempo. Cuando el robot vuelva a interrumpir la barrera, el tiempo se guardará como ``tiempo de vuelta''.
\end{itemize}    

Se reinicia la vuelta cuando ocurre alguno de los siguientes casos:
\begin{itemize}
  \item Si el robot invierte el sentido de avance.
  \item Si el robot se cae de la pista.
  \item Si el robot se desvía de la ruta planeada (por ejemplo, ante la presencia de cruces).
  \item Cualquier evento que requiera la intervención del equipo sobre el robot.
\end{itemize}    

\subsection*{Generales}
\begin{itemize}
  \item El robot no debe estar necesariamente sobre la línea, pero no debe desviarse de la ruta planeada.
  \item Si ningún equipo puede completar la vuelta al menos una vez, los organizadores podrán proponer un circuito más simple.
  \item Cada equipo es responsable de tener sus propias baterías y cargadores. Los equipos dispondrán de múltiples accesos a la red de energía eléctrica para cargar las baterías, pero no se otorgará tiempo extra para recargarlas cuando sean convocados a la pista, con lo cual es responsabilidad de cada grupo tener sus baterías cargadas.
\end{itemize}    

\subsection*{Puntuación}
Se calificará a los robots en tres categorías
\begin{enumerate}
  \item Resultado de la carrera (P1): se asignará un puntaje entre 0 y 10 puntos, según la siguiente escala:
  \begin{itemize}
    \item 10 puntos para el primero.
    \item  8 puntos para el segundo.
    \item  6 puntos para el tercero.
    \item  4 puntos para el cuarto.
    \item  2 puntos para los restantes robots que completaron una vuelta.
    \item  0 puntos para los que no completaron una vuelta del circuito.
  \end{itemize}    
    Se entiende por ``primero'' aquel robot que completa la vuelta en menor tiempo.

  \item Documentación (P2): El jurado calificará con una escala de 0 a 5 puntos la documentación presentada por cada equipo.
  A criterio del jurado se evaluará la calidad y completitud de todos los items indicados en la sección documentación.

  \item Originalidad del Robot (P3): El jurado calificará de 0 a 5 puntos según la originalidad que consideren de cada robot.
\end{enumerate}

Se dispondrá de una evaluación de cada jurado en relación a las categorías 2 y 3. Para cada una de estas categorías se promedian todos los puntajes otorgados por los jurado. Por ejemplo, para la categoría P2 se tendrán las notas de los 2 jurados. P2 se calcula como:

\begin{center}
P2=(J1+J2)/3
\end{center}

La puntuación final del robot se obtiene con la siguiente ecuación:

\begin{center}
Puntaje final = P1 $\cdot$ $0.5$ + P2 $\cdot$ $0.7$ + P3 $\cdot$ 0,3
\end{center}

Está ecuación junto con las escalas seleccionadas otorga pesos a cada categoría de la siguiente forma:

1) 50\%  2) 35\%  3) 15\%

La puntuación máxima posible es 10 puntos.

\textbf{El ganador de la competencia es el robot que obtenga mayor cantidad de puntos}.

------------------------------------------------------------------------------------------

Este reglamento fue confeccionado por los miembros del Club de Robótica utilizando como base los siguientes reglamentos:

\begin{enumerate}
  \item Grupo de Robótica UTN - Bahía Blanca - Competencia Velocistas: \\
    \url{http://www.grsbahiablanca.com.ar/compe_2012.htm}
  \item UC Davis Natcar: \url{http://www.ece.ucdavis.edu/natcar/rules.html}
\end{enumerate}
\end{document}

\documentclass[a4paper,12pt]{article}

\usepackage[utf8x]{inputenc}
\usepackage[spanish]{babel}
\usepackage[pdftex]{graphicx}

\usepackage[pdftex]{hyperref}
\usepackage{iwona}


\author{Ing. Lucas Chiesa \\ Club de Robótica \\ Facultad de Ingeniería \\ Universidad de Buenos Aires}
\title{Propuesta Desafío Nacional de Robótica}


\begin{document}

\maketitle

\begin{abstract}
 Esto documento tiene como objetivo proponer un desafío de robótica a nivel nacional que tenga para incentivar a los grupos de investigación del país, públicos o privados a trabajar en las técnicas más modernas y avanzadas de la robótica móvil. Para ello este desafío tiene que ser gran de gran dificultad, pero al mismo, proponer una meta alcanzable en el mediano plazo por los equipos de trabajo actuales.
\end{abstract}

\newpage
\section{Motivación}

Peter Norvig, actual director de investigación para Google y autor del libro de referencia en inteligencia artificial \textit{Artificial Intelligence: A Modern Approach} en una entrevista sobre el futuro de la robótica comentó:

\begin{quote}
\textit{In terms of robotics we’re probably where the world of PCs were in the early 1970s, where you could buy a PC kit and if you were an enthusiast you could have a lot of fun with that. Within a decade that changed, your grandmother needed word processing or email and we rapidly went from a very small number of hobbyists to pervasive technology throughout society in one or two decades. I expect a similar sort of timescale for robotic technology to take off, \textbf{starting roughly now}.}
\end{quote}

La industria informática, si bien mantiene un gran volumen de ventas y cada vez se expande más en la forma de los dispositivos móviles, al mismo tiempo, ya es una industria madura donde los principales jugadores ya están establecidos y no queda lugar para que una empresa nueva pueda posicionarse como un referente en el área. Mientras que los países asiáticos que históricamente fabricaban componentes electrónicos ahora pueden diseñar super computadora, los países occidentales están buscando el próximo gran mercado, que muchos, además de Norvig, piensan que es la robótica y la inteligencia artificial al servicio de la humanidad.

Como tal, es de sumo interés para nuestro país no perder esta nueva oportunidad. Ahora es el momento de desarrollar conocimiento, tecnología y aplicaciones de la robótica para así ayudar a desarrollar este nuevo mercado, no sólo en el país sino en el mundo. Ayudar a crear este nuevo mercado significa la oportunidad única de posicionar al país al frente de estas nuevas tecnologías desde el comienzo.
 
Actualmente existen diversos grupos de investigación de diferentes universidades dedicados a la robótica, sin embargo la mayoría de estos grupos son relativamente recientes y el nivel de producción está muy lejos de otras áreas de investigación como pueden ser las ciencias médicas o la biología molecular.

Es por esto que se propone organizar un desafío nacional de robótica, cuya dificultad sea superior al estado del arte de la robótica en el país, y así fomentar el crecimiento de los actuales grupos de trabajo, como también la formación de nuevos grupos, especialmente en el ámbito privado, que actualmente permanece alejado de esta temática. Este desafío debe contar con premios acordes con la dificultad del mismo, y se deberían diseñar mecanismos de financiamiento para que los participantes puedan afrontar los costos de desarrollar un robot para esta competencia.

Esta competencia serviría como puntapié de una campaña nacional de apoyo e incentivo de las actividades del sector.

\end{document}

\section{Antecedentes}


